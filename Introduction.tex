\newcommand{\introGAKE}{\text{GAKE}\xspace}
\newcommand{\introAKE}{\text{AKE}\xspace}
\newcommand{\introNumSess}{s}
\newcommand{\introNumUser}{n}


\section{Introduction}\label{sec:introduction}
<<<<<<< HEAD
A group authenticated key exchange (\introGAKE) protocol allows a group of users to agree on a shared session key in the present of an active adversary. It extends the \introAKE protocol from 2-party to $n$-party setting. Similar to the security requirements of 
=======
%<<<<<<< HEAD
A group authenticated key exchange (\introGAKE) protocol allows a group of users to agree on a shared session key. It extends the \introAKE protocol from 2-party to $n$-party setting. Similar to the security requirements of \introAKE protocols, we allow adversaries to have full control of the communication, and they can also reveal some of the shared session keys and corrupt long-term secret key of some honest users. 
In the end, we require that the shared session key from a \introGAKE should be pseudo-random.

\paragraph{Existing Constructions.}
Constructing secure \introGAKE protocols has a long history, and almost all the existing protocols \cite{CCS:BCPQ01,AC:BreChePoi01,EC:BreChePoi02,PQCRYPTO:ADGK19,JC:PanQiaRin22} are constructed via a signature-based approach. Namely, it uses a digital signature scheme to authenticate a passively secure group key protocol. Such a signature-based approach is rather inefficient. We take the protocol in \cite{JC:PanQiaRin22} as an example: To agree on a session key with a group of $n$ users, each user needs to run the signing algorithm 2 times and the verification algorithm $3(n-1)$ times. With Schnorr's signature scheme, this already requires $2+6(n-1)$ times group exponentiation. Taking the security loss of Schnorr's signature into account, this can be very inefficient in practice. Hence, we are interested in constructing a secure \introGAKE without signatures in this paper.

To the best of our knowledge, the work of Li and Yang \cite{CANS:LiYan13} is the only signature-less \introGAKE protocol. However, it requires the use of (cryptographic) multilinear maps \cite{EC:GarGenHal13}, which is a rather strong requirement. Until now, we do not know any efficient multilinear maps. This leads to the first goal of our work, namely, to construct a \textit{signature-less} \introGAKE protocol from \textit{a weak, standard assumption.}

%\begin{displayquote}
%	\emph{Our First Goal:} Constructing a secure \introGAKE protocol from a weak, standard assumption.
%\end{displayquote} 

% They all have single-Test query

\paragraph{Tight Security.}
Moreover, most of the aforementioned protocols have large security loss. For instance, the protocol of Bresson, Chevassute, and Pointcheval \cite{AC:BreChePoi01} has security loss $O(\introNumUser \cdot \introNumSess)$, where $\introNumUser$ and $\introNumSess$ are the numbers of users and sessions per user, respectively. Even worse, protocols in \cite{CCS:BCPQ01,CANS:LiYan13} lose at least an exponential factor $\introNumSess^{\introNumUser}$. This makes it particularly inefficient to instantiate these protocols in a theoretically sound manner.
%Say all existing protocols have large security loss

\jiaxin{Explain tight security, etc...}

\paragraph{``Sweet-spot'' between Efficiency and Tightness.}
We are interested in tightness, but more into searching the sweet-spot between tightness and efficiency.
%=======
%A group authenticated key exchange %(\introGAKE) protocol allows a group of users to agree on a shared session key in the present of an active adversary. It extends the \introAKE protocol from 2-party to $n$-party setting. Similar to the security requirements of 
%>>>>>>> cbdb7a8770ca4858481090c54c9d65cf24400923
>>>>>>> 35069758ab56cb071f65bc6b9a04346080ec6cbe

\subsection{Our Contribution}
How magnificent have we been?

Our efficiency $\approx$ $3\times $ BD, but much less than the Sign-GAKE (2*Sign + 3(n-1)*Ver) per user.

\paragraph{Technical Overview: A Simple Twist on BD.}

\paragraph{Open Problems.}
We leave constructing an efficient signature-less post-quantum \introGAKE protocol as an interesting open problem. 
Standard model construction?

%\subsection{Related Work.}
%There will be a lot to write about here, at least comparing efficiency.

\subsection{Roadmap.}
Going through the paper highlighting section by section.
