\newcommand{\introGAKE}{\text{GAKE}\xspace}
\newcommand{\introAKE}{\text{AKE}\xspace}
\newcommand{\introNumSess}{s}
\newcommand{\introNumUser}{n}
\newcommand{\reduction}{\mathcal{R}}


\section{Introduction}\label{sec:introduction}

A group authenticated key exchange (\introGAKE) protocol allows a group of users to agree on a shared session key. It extends the \introAKE protocol from 2-party to $n$-party setting. Similar to the security requirements of \introAKE protocols, we allow adversaries to have full control of the communication, and they can also reveal some of the shared session keys and corrupt long-term secret key of some honest users. 
In the end, we require that the shared session key from a \introGAKE should be pseudo-random.

\paragraph{Existing Constructions.}
Constructing secure \introGAKE protocols has a long history, and almost all the existing protocols \cite{CCS:BCPQ01,AC:BreChePoi01,EC:BreChePoi02,PQCRYPTO:ADGK19,JC:PanQiaRin22} are constructed via a signature-based approach. Namely, it uses a digital signature scheme to authenticate a passively secure group key protocol. Such a signature-based approach is rather inefficient. We take the protocol in \cite{JC:PanQiaRin22} as an example: To agree on a session key with a group of $n$ users, each user needs to run the signing algorithm 2 times and the verification algorithm $3(n-1)$ times. With Schnorr's signature scheme, this already requires $2+6(n-1)$ times group exponentiation. Taking the security loss of Schnorr's signature into account, this can be very inefficient in practice. Hence, we are interested in constructing a secure \introGAKE without signatures in this paper.
We stress that this is not only relevant for efficiency, but also interesting in theory to consider  whether it is possible to construct a secure \introGAKE protocol without signatures.

To the best of our knowledge, the work of Li and Yang \cite{CANS:LiYan13} is the only signature-less \introGAKE protocol. However, it requires the use of (cryptographic) multilinear maps \cite{EC:GarGenHal13}, which is a rather strong requirement. Until now, we do not know any efficient multilinear maps. 
This leads to the first goal of our work, namely:
%This leads to the first goal of our work, namely, to construct a \textit{signature-less} \introGAKE protocol from \textit{a weak, standard assumption.}

\begin{displayquote}
	\emph{Our First Goal:} Constructing a secure \introGAKE protocol from a weak, standard assumption.
\end{displayquote} 

% They all have single-Test query

\paragraph{Tight Security.}
We observe that most of the aforementioned protocols have large security loss. For instance, the protocol of Bresson, Chevassute, and Pointcheval \cite{EC:BreChePoi02} has security loss $O(\introNumUser \cdot \introNumSess)$, where $\introNumUser$ and $\introNumSess$ are the numbers of users and sessions per user, respectively. Even worse, protocols in \cite{CCS:BCPQ01,CANS:LiYan13} lose at least an exponential factor of $\introNumSess^{\introNumUser}$.
This makes it particularly inefficient to instantiate these protocols in a theoretically sound manner.
%Say all existing protocols have large security loss

Security loss is a concrete factor to quantify security. More precisely, when we prove the security of a protocol $\Pi$, we construct a reduction $\reduction$ to show that if an adversary can break the security of $\Pi$ with probability $\varepsilon_{\Pi}$, then $\reduction$ can break the underlying computational problem $P$ with probability $\varepsilon_{P} \geq \nicefrac{\varepsilon_{\advA}}{\ell}$, and we assume the runtime of $\reduction$ is roughly the same as $\advA$.. 
Here the factor $\ell$ is the security loss, 
When $\ell$ is a small constant, we call the security reduction tight. 

Among the known \introGAKE protocols, the work of Pan, Qian, and Ringerud (PQR) \cite{JC:PanQiaRin22} is the only tight construction that we know, but it requires a tightly secure signature scheme with adaptive corruptions. The most efficient construction of such a signature is due to Diemert, Gellert, Jager, and Lyu (DGJL) \cite{PKC:DGJL21}, which 
%contains 3 $\ZZ_p$-elements in the signature and 
requires 6 exponentiation for signing a message and 8 exponentiation for verifying a signature. Asymptotically, this is rather inefficient, considering that Schnorr's signature only need 1 exponentiation for signing and 2 exponentiation for verification.
%Hence, it is interesting to consider whether there is a more efficient  
%\jiaxin{Explain tight security, etc...}

\paragraph{A Balanced View on Tight Security.}
It is usually desirable to have tight security proofs. 
This is because, when instantiating a non-tight protocol $\Pi$ according to the security proof, one needs to increase the security parameter to compensate the security loss $\ell$.
\jiaxin{I am here.}

We are interested in tightness, but more into searching the sweet-spot between tightness and efficiency.


\subsection{Our Contribution}
How magnificent have we been?

Our efficiency $\approx$ $3\times $ BD, but much less than the Sign-GAKE (2*Sign + 3(n-1)*Ver) per user.

\paragraph{Technical Overview: A Simple Twist on the Burmester-Desmedt Protocol.}

\paragraph{Open Problems.}
We leave constructing an efficient signature-less post-quantum \introGAKE protocol as an interesting open problem. 
Standard model construction?

%\subsection{Related Work.}
%There will be a lot to write about here, at least comparing efficiency.

\subsection{Roadmap.}
Going through the paper highlighting section by section.
