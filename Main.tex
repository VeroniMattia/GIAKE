\documentclass[runningheads,envcountsame]{llncs}
\usepackage{amsmath,amsfonts, amssymb, mathtools,stackrel}
\usepackage[usenames]{color}
\usepackage[dvipsnames]{xcolor}
\usepackage{colortbl}
\usepackage{algorithm}
\usepackage[noend]{algpseudocode}
\usepackage[lambda, advantage ,operators, sets,adversary,landau,probability,notions,logic,ff,mm,primitives,events,complexity,asymptotics,keys]{cryptocode}
\usepackage{comment}
\usepackage{numprint}
\usepackage{mdframed}
\npthousandsep{\,}
\usepackage{tikz-cd}
\definecolor{NTNUBlue}{HTML}{00509e}
\usepackage[bookmarks,bookmarksdepth=2,pdfusetitle,colorlinks,linkcolor=NTNUBlue,citecolor=NTNUBlue,urlcolor=NTNUBlue]{hyperref}

\usepackage[nameinlink]{cleveref}
\usepackage{multirow}

\usepackage{todonotes}

 \spnewtheorem{plaindef}[theorem]{Definition}{\bfseries}{}
% Comments
\newcommand{\mattia}[1]{\hspace*{0,01pt}\todo[color=NTNUBlue!40]{#1}}

% Generic math
\newcommand{\N}{\mathbb{N}}
\newcommand{\Prob}{{Pr}}

\bibliographystyle{splncs03}
\usepackage{ulem}
\usepackage{nicodemus}
\usepackage{xspace}
\usepackage{stmaryrd}
\usepackage{mathtools}
\usepackage{dashbox}
\usepackage{xparse}
\usepackage{xargs}
\usepackage{struktex}
\usepackage{upgreek}
\usepackage{subcaption}
\usepackage{adjustbox}
\setlength\intextsep{10pt}
%================================
%=========  MAIN BODY ===========
%================================
\begin{document}
	
	\title{Session-Tight Group Authenticated Key Exchange from Weak Assumptions}
	
	\titlerunning{Short paper title}
	
	
	
	\author{Jiaxin Pan\inst{1}
		and Mattia Veroni\inst{2}\thanks{Supported by Research Council of Norway under Project No. 32423}}
	\authorrunning{J. Pan, M. Veroni}
	
	\institute{University of Kassel, Kassel, Germany \\
		\email{}
		\and  NTNU - Norwegian University of Science and Technology, Trondheim, Norway \\
		\email{mattia.veroni@ntnu.no}}
	
	\maketitle           
	
	\raggedbottom
	\begin{abstract}
		We construct an efficient group authenticated key exchange protocol from weak assumptions in the random oracle model. Compared with the signature-based protocol (Pan, Qian, and Ringerud, Journal of Cryptology 2022) and the multilinear-map-based protocol (Li and Yang, CANS 2013), our protocol does not require any signature and has security based on the Diffie-Hellman assumption without pairings. 
		
		\quad Our protocol can be viewed as a multiparty variant of the \textsf{X3DH} protocol. It is session-tight, meaning the security of our protocol is independent of the number of sessions, but dependent linearly of the number of users. Compared with the signature-based protocol of Pan et al., although our protocol is only session-tight, it is X \% more efficient.
		\jiaxin{I will updated it after the efficiency table is done.}
		
		\keywords{Group authenticated key exchange \and Tightness \and Random oracles}
	\end{abstract}
	
	\setlength{\tabcolsep}{3pt}
	\renewcommand{\arraystretch}{1.15}
	
	\newcommand{\introGAKE}{\text{GAKE}\xspace}
\newcommand{\introAKE}{\text{AKE}\xspace}
\newcommand{\introNumSess}{s}
\newcommand{\introNumUser}{n}
\newcommand{\reduction}{\mathcal{R}}
\newcommand{\CCGJJ}{CCGJJ\xspace}


\section{Introduction}\label{sec:introduction}

A group authenticated key exchange (\introGAKE) protocol allows a group of users to agree on a shared session key. It extends the \introAKE protocol from 2-party to $n$-party setting. Similar to the security requirements of \introAKE protocols, we allow adversaries to have full control of the communication, and they can also reveal some of the shared session keys and corrupt long-term secret key of some honest users. 
In the end, we require that the shared session key from a \introGAKE should be pseudo-random.

\paragraph{Existing Constructions.}
Constructing secure \introGAKE protocols has a long history, and almost all the existing protocols \cite{CCS:BCPQ01,AC:BreChePoi01,EC:BreChePoi02,PQCRYPTO:ADGK19,JC:PanQiaRin22} are constructed via a signature-based approach. Namely, it uses a digital signature scheme to authenticate a passively secure group key protocol. Such a signature-based approach is rather inefficient. We take the protocol in \cite{JC:PanQiaRin22} as an example: To agree on a session key with a group of $n$ users, each user needs to run the signing algorithm 2 times and the verification algorithm $3(n-1)$ times. With Schnorr's signature scheme, this already requires $2+6(n-1)$ times group exponentiation. Taking the security loss of Schnorr's signature into account, this can be very inefficient in practice. Hence, we are interested in constructing a secure \introGAKE without signatures in this paper.
We stress that this is not only relevant for efficiency, but also interesting in theory to consider  whether it is possible to construct a secure \introGAKE protocol without signatures.

To the best of our knowledge, the work of Li and Yang \cite{CANS:LiYan13} is the only signature-less \introGAKE protocol. However, it requires the use of (cryptographic) multilinear maps \cite{EC:GarGenHal13}, which is a rather strong requirement. Until now, we do not know any efficient multilinear maps. 
This leads to the first goal of our work, namely:
%This leads to the first goal of our work, namely, to construct a \textit{signature-less} \introGAKE protocol from \textit{a weak, standard assumption.}

\begin{displayquote}
	\emph{Our First Goal:} Constructing a secure \introGAKE protocol from \textit{weak, standard assumptions}.
\end{displayquote} 

% They all have single-Test query

\paragraph{Tight Security.}
We observe that most of the aforementioned protocols have large security loss. For instance, the protocol of Bresson, Chevassute, and Pointcheval \cite{EC:BreChePoi02} has security loss $O(\introNumUser \cdot \introNumSess)$, where $\introNumUser$ and $\introNumSess$ are the numbers of users and sessions per user, respectively. Even worse, protocols in \cite{CCS:BCPQ01,CANS:LiYan13} lose at least an exponential factor of $\introNumSess^{\introNumUser}$.
This makes it particularly inefficient to instantiate these protocols in a theoretically sound manner.
%Say all existing protocols have large security loss

Security loss is a concrete factor to quantify security. More precisely, when we prove the security of a protocol $\Pi$, we construct a reduction $\reduction$ to show that if an adversary can break the security of $\Pi$ with probability $\varepsilon_{\Pi}$, then $\reduction$ can break the underlying computational problem $P$ with probability $\varepsilon_{P} \geq \nicefrac{\varepsilon_{\advA}}{\ell}$, and we assume the runtime of $\reduction$ is roughly the same as $\advA$.. 
Here the factor $\ell$ is the security loss, 
When $\ell$ is a small constant, we call the security reduction tight. 

Among the known \introGAKE protocols, the work of Pan, Qian, and Ringerud (PQR) \cite{JC:PanQiaRin22} is the only tight construction that we know, but it requires a tightly secure signature scheme with adaptive corruptions. The most efficient construction of such a signature is due to Diemert, Gellert, Jager, and Lyu (DGJL) \cite{PKC:DGJL21}, which 
%contains 3 $\ZZ_p$-elements in the signature and 
requires 6 exponentiation for signing a message and 8 exponentiation for verifying a signature. Asymptotically, this is rather inefficient, considering that Schnorr's signature only needs 1 exponentiation for signing and 2 exponentiation for verification.
We aim to further lower the cost for tightness.
%Hence, it is interesting to consider whether there is a more efficient  
%\jiaxin{Explain tight security, etc...}

\paragraph{A Balanced View on Tight Security.}
It is usually desirable to have tight security proofs. 
This is because, when instantiating a non-tight protocol $\Pi$ according to the security proof, one needs to increase the security parameter to compensate the security loss $\ell$.
However, when it costs much more exponentiation to achieve tight security, it will lower its practical relevance, since it can lead to much worse concrete efficiency than a non-tight protocol. Hence, our goal is to achieve better concrete efficiency with a balance view on tightness, and we search a ``sweet-spot'' between asymptotic efficiency and tight security.

%We are interested in tightness, but more into searching the sweet-spot between tightness and efficiency.

This is inspired by the work of Cohn-Gordon et al. (\CCGJJ) \cite{C:CCGJJ19}, where the security loss of their \introAKE protocol is independent of the number of sessions, but linearly dependent of the number of users. We call such a protocol \textit{session-tight} for simplicity. Although the \CCGJJ protocol is not fully tight (but session-tight), it achieves better efficiency than its fully tight counter-part. According to \cite[Fig. 2]{EC:JKRS21}, the communication complexity of \CCGJJ is less than half of the fully tight protocol, $\mathsf{AKE_{wFS,DDH}}$. In terms of computation complexity, the \CCGJJ protocol only requires 4 exponentiation per users to agree on a session key, compared to at least 6 per users in $\mathsf{AKE_{wFS,DDH}}$.
Hence, our second goal is

\begin{displayquote}
	\emph{Our Second Goal:} Constructing an \textit{efficient} \introGAKE protocol via \textit{session-tight} reductions.
\end{displayquote} 


\subsection{Our Contribution}
How magnificent have we been?

Our efficiency $\approx$ $3\times $ BD, but much less than the Sign-GAKE (2*Sign + 3(n-1)*Ver) per user.

\paragraph{Technical Overview: A Simple Twist on the Burmester-Desmedt Protocol.}

\paragraph{Open Problems.}
We leave constructing an efficient signature-less post-quantum \introGAKE protocol as an interesting open problem. 
Standard model construction?

%\subsection{Related Work.}
%There will be a lot to write about here, at least comparing efficiency.

\subsection{Roadmap.}
Going through the paper highlighting section by section.

	\section{Preliminaries}\label{sec:preliminaries}
Where we put all the cryptographic preliminaries.

\subsection{Group Implicitly Authenticated Key Exchange}\label{subsec:GIAKE}
\jiaxin{I probably won't use the term "implicitly authenticated" because people are very obsessed with these explicit vs. implicit authentication and have strong opinions on it, but just state authenticated and define weak forward secrecy for it.}
\mattia{I will specify that we mean entity authentication, and by means of only the key exchange algorithms and no other extra mechanism or assumption}
A Group Key Exchange (GKE) protocol allows a group of parties to agree on a shared secret key. 
If the key exchange is authenticated, meaning that the entities are somehow authenticated after the protocol run, we talk about Group Authenticated Key Exchange (GAKE); if authentication is achieved without the use of any additional cryptographic primitive (e.g.digital signatures or massage authentication codes), we say that the protocol is implicitly authenticated (and thus GIAKE).

\paragraph{Terminology.} Let us first recall some basic terminology for group key-exchange (GKE) protocols.
A global \textit{session} is an execution of the protocol, consisting of all algorithmic runs (inputs and outputs) and messages exchanged between all parties, resulting in the final computation of the \textit{group key}.
A local \textit{instance} is what a participant computes, influenced by the received and sent messages.
The local instance is run in a \textit{context}, which consists of all the information that does not come directly from algorithmic outputs or received messages, e.g. set of participants, number of sessions, etc.
We call a participant in a GKE session a \textit{party}.

\paragraph{The generic protocol.} We consider the case of an interactive protocol, run by a group $\PartySet = (\Party_1,\allowbreak \Party_2,\dots,\Party_n)$ of $n$ parties, with $2 < n \leq \projectKey$, arranged in a cycle (operations on the indices are taken modulo the group size).
Each party $\User[i]$ holds a static (long-term) key pair $(\ssk_i,\spk_i)$; we assume the existence of a Public-Key Infrastructure (PKI) that allows for public-key look-up.
In particular, we let PKI register the party identities according to some ordering (e.g. time of registration to the PKI), implicitly determining how to sort the participants to a key-exchange session in a sequence.

\begin{plaindef}[GIAKE]\label{def:GKE}
	Let $\projectKey \in \NN$ be the maximum number of parties, and let $\secparam\in \NN$ be the security parameter. 
	Let $\PartySet = (\User[1],\User[2],\dots,\allowbreak \User[n])$ be a list of $n \leq \projectKey$ parties that want to establish a shared secret key.
	A \textit{Group Implicitly Authenticated Key Exchange} (GIAKE) protocol is a tuple of four algorithms $\GIAKE = (\KeyGen, \GAKEIni, \GAKERes, \GAKEDer)$ defined as follows:
	\begin{itemize}
		\item $(\sk,\pk) \getsr \KeyGen(1^\secpar)$: a PPT key-generation algorithm that outputs a uniformly random secret key $\sk$ and the corresponding public key $\pk$.
		Each party $\User[i]$ preventively produces a static key pair $(\ssk_i,\spk_i) \getsr \KeyGen(1^\secpar)$ and registers the public key $\spk_i$ to the PKI;
		\item $(m_i,\st) \getsr \GAKEIni (\ssk_i,(\spk_j)_{1\leq j \leq n})$: a PPT session initialisation algorithm that, on input a party's static secret key $\ssk_i$ and the static public keys of the intended participants in the session, outputs a message $\messageIni$ and a state $\st$;
		\item $(\hat{m}_i,\st) \gets \GAKERes(\ssk_i,\st,(\spk_j,m_j)_{1\leq j \leq n})$: a (possibly PPT) algorithm that, on input a party's static secret key $\ssk_i$, a state $\st$ and the messages $m_j$ (each broadcasted by the party with static public key $\spk_j$ after running the $\GAKEIni$ algorithm), outputs a message $\hat{m}_i$ and updates the state $\st$;
		\item $K \gets \GAKEDer (\spk_i,\st,(\spk_j,m_j,\hat{m}_j)_{1\leq j \leq n})$: a deterministic algorithm that,  on input a party's static secret key $\ssk_i$, a state $\st$ and the messages $(m_j,\hat{m}_j)$ output by each party $\User[j]$ with static public key $\spk_j$, derives a session key.
	\end{itemize}
	The scheme must implicitly guarantee authentication, i.e. without the use of other primitives.
\end{plaindef}	

\begin{figure}
	\centering 
	
	\begin{tikzpicture}
		\matrix (m)[matrix of nodes, column  sep=1em,row  sep=1mm,%
		column 2/.style={minimum width={width(" $(\spk_j,m_j,\hat{m}_j)_{0\leq j \leq n-1,j\neq i}$")},anchor=center},% 
		column 3/.style={minimum width=6em,anchor=center} ]{
			%line 1
			\underline{$\User[i](\ssk_i, \spk_i)$}& & \underline{$\PartySet\setminus\{\User[i]\}$} \\[2mm]
			%line 2
			$(m_i,\st) \getsr \GAKEIni (\ssk_i,(\spk_j)_{0\leq j \leq n-1})$ & &\\
			%line 3
			& $(\spk_i,m_i)$ & \\
			%line 4
			& $(\spk_j,m_j)_{0\leq j \leq n-1,j\neq i}$ & \\
			%line 5
			$(\hat{m}_i,\st) \gets \GAKERes(\ssk_i,\st,(\spk_j,m_j)_{0\leq j \leq n-1})$ & & \\ 
			%line 6
			& $(\spk_i,\hat{m}_i)$ & \\
			%line 7
			& $(\spk_j,m_j,\hat{m}_j)_{0\leq j \leq n-1,j\neq i}$ & \\
			%line 8
			$K \gets \GAKEDer (\spk_i,\st,(\spk_j,m_j,\hat{m}_j)_{0\leq j \leq n-1})$ & & \\ 
		};
		% drawing arrows
		\draw[-latex] (m-3-2.south west)--(m-3-2.south east);
		\draw[-latex] (m-4-2.south east)--(m-4-2.south west);
		\draw[-latex] (m-6-2.south west)--(m-6-2.south east);
		\draw[-latex] (m-7-2.south east)--(m-7-2.south west);
		\draw (m-8-1.south west) rectangle (m-1-3.north east);
	\end{tikzpicture}
	\caption{A $\GIAKE$ protocol flow from party $\User[i]$'s point of view. All messages (both sent and received ones) are broadcasted to all parties.}\label{fig:giakeprocedure}
\end{figure}

	We now describe a protocol run as viewed by party $\User[i]$, an illustration of which is given in \Cref{fig:giakeprocedure}.
	In the setup phase, each party $\User[i]$ generates a static key-pair $(\ssk_i,\spk_i) \gets \KeyGen(1^\secpar)$ and registers the public part $\spk_i$ to the PKI.
	In the first round, each party runs the initialisation algorithm \[(m_i,\st) \getsr \GAKEIni (\ssk_i,(\spk_j)_{1\leq j \leq n})\] on input its static secret key and the intended parties' public keys. 
	The resulting outputs are an internal state $\st$ and a message $m_i$, the latter of which is broadcasted together with $\spk_i$.
	Upon retrieving $\{(\spk_j,m_j)\}_{1\leq j \leq n, j \neq i}$ from the broadcast channel, each party runs the response algorithm \[(\hat{m}_i,\st) \gets \GAKERes(\ssk_i,\st,(\spk_j,m_j)_{1\leq j \leq n})\], updating its internal state and broadcasting $(\spk_i,\hat{m}_i)$.
	In the last phase, each party derives the group key for the session by running \[K \gets \GAKEDer (\spk_i,\st,(\spk_j,m_j,\hat{m}_j)_{1\leq j \leq n})\].
	
	In broad terms, a GIAKE protocol must be \textit{correct} (each honest party can successfully compute the group key at the end of an honest session), \textit{implicitly authenticated} (the four algorithms in $\GIAKE$ are sufficient for each party to rest assured that nobody but the intended participants may gain access to the session key) and secure (the meaning of which varies; see \Cref{subsec:secmodel}).

\subsection{Security model for GIAKE}\label{subsec:secmodel}
We now describe a security model for a two-round broadcast group authenticated key exchange, that allows a set of $n > 2$ parties to establish a common secret key.
The adversarial model is borrowed from \cite[Section 6.1]{PQR22}, which is in itself an extension to $\projectKey$ parties of the model described in \cite{JKRS20}.
Since we aim for implicit authentication for our 2-round protocol, we have to content ourselves with the notion of \textit{weak Forward Secrecy} (wFS), since no unsigned 2-round key-exchange protocol can hope for (full) Forward Secrecy \cite[Section 3.2]{HMQV}.
In particular, w.r.t. \cite[Section 6.1]{PQR22} we do not allow the adversary to actively participate in the session.\mattia{Argument: the adversary can act like party $\User[i]$ sending an ephemeral key of its choice, since it is unauthenticated. Then it reveals the long-term key of $\User[i]$ and computes the session key.}

In this security model, no mechanism to provide explicit message authentication is considered.
Instead, implicit authentication implies that only the intended honest parties can successfully compute the group key for the session.

\begin{figure}[h!]
	\centering
	\scalebox{0.90}{
		\fbox{\small
			\begin{minipage}[t]{7.5cm}
				\underline{\textbf{GAME} $\GIAKEsecurityPFSGame$} 
				\begin{nicodemus}[1]
					\item $\pcfor n \in [\mu]$
					\item \quad $(\pk_n, \sk_n) \leftarrow \KeyGen(\secparam)$
					\item $b\getsr \{0,1\}$
					\item $b' \gets	\Adv^{O}(\pk_1, \cdots, \pk_\mu)$
					\item \pcfor $\sID^*\in\testSessions$:
					\item \quad \pcif $\FreshnessO(\sID^*)=\pcfalse$
					\item \quad \quad $\pcreturn 0$\gcom{session not fresh}\label{line:gakefreshness}
					\item \quad \pcif $\ValidAttackO(\sID^*)=\pcfalse$
					\item \quad \quad  $\pcreturn 0$\gcom{no valid attack}
					\item $\pcreturn \bool{b=b'}$ \\
				\end{nicodemus}
				\underline{$\GakeIniO ( \GIni  \in [\mu], \PartySet_{i} \subseteq [\mu])$}
				\begin{nicodemus}
					\item $\cntS~\incr$
					\item $\sID \coloneqq \cntS$
					\item $\owner[\sID]:=\GIni $ 
					\item $\peer[\sID]:=\PartySet{i} $
					\item $\GakeUSet[\sID] \coloneqq \peer[\sID] \cup \{\GIni\}$
					\item $(\GakeIniMes_{i}, \Gstate) \getsr \GakeIni(\GakeSk[{i}], \{\GakePk[j]\}_{j \in \PartySet{i}})$
					\item $\GIniMes[\sID]:= ({i},\GakeIniMes_{i}) $
					\item $\Gstates[\sID]:=\Gstate$
					\item $\pcreturn (\sID, \GakeIniMes_{i})$\\
				\end{nicodemus}

				\underline{$\GakeResO(\sID \in [\cntS], \GakeIniMesSet[{i}] )$} %\gcom{$\GakeIniMesSet[{i}]$ is a set of messages}
				\begin{nicodemus}
					\item $(\GIni,\PartySet{i}) := (\owner[\sID],\peer[\sID])$
					\item \pcif $|\GakeIniMesSet[{i}]| \neq |\PartySet{i}|$ 
					\item \quad  \pcreturn $\bot$ 
					\gcom{all peers must have broadcasted}
					\item $\peercorrupted[\sID] := \bigvee\limits_{ j \in \PartySet{i}} \corrupted[j]$
					\item $\GIniMesSet[\sID] \coloneqq  \GakeIniMesSet[{i}]$ \label{line:gake-problem-a}
					\item $(\GakeResMes_{i}, \Gstate) \getsr \GakeRes(\GakeSk[\GIni], \{\GakePk[j]\}_{j \in \PartySet{i}}, \Gstates[\GakeResUsID], \GakeIniMesSet[{i}])$
					\item $\GResMes[\sID]:=({i},\GakeResMes_{i})$
					\item $\Gstates[\sID]:= \Gstate$
					\item \pcreturn $\GakeResMes_{i}$ 
				\end{nicodemus}
			\end{minipage}
			\begin{minipage}[t]{7cm}
				\underline{$\GakeDerO(\GakeResUsID \in [\cntS],  \GakeResMesSet[{i}])$} 
				\begin{nicodemus}
					\item \pcif $\skeys[\GakeResUsID] \neq \bot$  
					\item \quad \pcreturn $\bot$
					\item $({i},\PartySet{i}) := (\owner[\GakeResUsID],\peer[\GakeResUsID])$
					\item \pcif $|\GakeResMesSet[{i}]| \neq |\PartySet{i}|$ \pcreturn $\bot$
					\item $\peercorrupted[\sID] := \bigvee\limits_{ j \in \PartySet{i}} \corrupted[j]$
					\item $\GResMesSet[\sID] \coloneqq \GakeResMesSet[{i}]$ \label{line:gake-problem-b}
					\item $\GakeIniMesSet[{i}] \coloneqq \GIniMesSet[\sID]$
					\item $\GakeSessKey \coloneqq \GakeDer(\GakeSk[{i}], \{\GakePk[j]\}_{j \in \PartySet{i}}, \Gstates[\GakeResUsID], \GakeIniMesSet[{i}], \GakeResMesSet[{i}])$
					\item $\skeys[\GakeResUsID] :=  \GakeSessKey$
					\item \pcreturn $\emptystr$\\
				\end{nicodemus}
				\underline{$\RevealO(\sID)$}
				\begin{nicodemus}
					\item $\revealed[\sID] \coloneqq \true$
					\item $\pcreturn \skeys[\sID]$	\\	
				\end{nicodemus}
				
				\underline{$\CorruptO(n \in [\mu])$}
				\begin{nicodemus}
					\item $\corrupted[n] := \true$
					\item $\pcreturn \sk_n$\\
				\end{nicodemus}
				
				\underline{$\TestO(\sID$)}
				\begin{nicodemus}
					\item $\pcif \sid\in\testSessions~ \pcreturn \bot$ \gcom{already tested}
					\item \pcif $\skeys[\sID] = \bot$~\pcreturn $\bot$
					\item $\testSessions \coloneqq \testSessions \cup \{\sid\}$
					\item $K_0^* \coloneqq \skeys[\sID]$
					\item $K_1^* \getsr \mathcal{K}$
					\item \pcreturn $K_b^*$
				\end{nicodemus}
			\end{minipage}
	}}
	\caption{
	Game $\GIAKEsecurityPFSGame$ for $\GIAKE$.
	The number of messages in the set $\GakeIniMesSet[\GakeUId]$ is denoted by $|\GakeIniMesSet[\GakeUId]|$, and 
	$|\peers_\GakeUId|$ denotes the number of parties in $\peers_\GakeUId$.
	\label{fig:game-ind-atk-gake}}
\end{figure}


%\heading{Execution environment.} We consider $\mu$ parties $\PartySet = (\User[1], \ldots, \User[\mu])$ with long-term key pairs $(\pk_i, \sk_i), i \in [\mu]$. For each group key exchange, each party in a group $\GakeUSet$ has their own session with a unique identification number $\sID$, and variables which are defined relative to $\sID$:
%
%\begin{itemize}
%	\item $\owner[\sID] \in [\mu]$ denotes the owner of the session.
%	\item $\peer[\sID] \subseteq[\mu]$ denotes the peers of the session.
%	\item $\GakeUSet[\sID]$ denotes all the participants of the session.
%	%	\item $\GakeUSet[\sID]$ denotes all the group members. %(i.e. $\peer[\sID]\cup \{\owner[\sID]\}$).
%	\item $\GIniMes[\sID]$ denotes the message sent by the owner during the first round.
%	\item $\GIniMesSet[\sID]$ denotes the messages received by the owner during the first round.
%	\item $\GResMes[\sID]$ denotes the message sent by the owner during the second round.
%	\item $\GResMesSet[\sID]$ denotes the messages received by the owner during the second round.
%	\item $\Gstates[\sID]$ denotes the (secret) state information {\ie} ephemeral secret keys.
%	\item $\ListGakeSessKey[\sID]$ denotes the session key.
%\end{itemize}
%
%\heading{Adversary model.} Similar to the $\AKE$ security notion, we do not allow the adversary to register adversarially controlled parties by providing long-term public keys, and the adversary has access to oracles $\CorruptO$ and $\RevealO$ as described in ~\cref{fig:game-ind-atk-gake}. We use the following boolean values to store which queries the adversary made:
%
%\begin{itemize}
%	\item $\corrupted[{i}]$ denotes whether the long-term secret key of party $\User[{i}]$ was given to the adversary.
%	\item $\revealed[\sID]$ denotes whether the group session key was given to the adversary.
%	\item $\peercorrupted[\sID]$ denotes whether one of the peers in the group was corrupted and its long-term key was given to the adversary at the time when the session key was derived.
%\end{itemize}
%
%
%\heading{Matching sessions.} Extending the definition of matching sessions from the two-party case, we define matching sessions in the $\Gake$ setting as follows.
%%\begin{itemize}
%%	\item \textbf{Matching Sessions}: Given a set of parties $\GakeUSet$ participating in a $\Gake$ session, the set of sessions 
%%	$\{\sID_j\}_{j \in |\GakeUSet |}$ are matching, if there exists $\GakeIniMesSet[{i}], \GakeResMesSet[{i}]$, such that for each $\User[j] \in \GakeUSet$ we have:
%%	$\owner[\sID_j] = \User[j]$, $\peer[\sID_j] = \GakeUSet\setminus\{j\}$, $\GIniMes[\sID_j] = \GakeIniMesSet[{i}]$ and $\GResMes[\sID_j] = \GakeResMesSet[{i}]$. (Notice that, we assume $\GakeUSet$ is an ordered set and $\User[j]$ represents the $j$-th party in $\GakeUSet$. The index $j$ is different from the party's index in the set of all participants $\PartySet$.)
%%\end{itemize}
%\begin{itemize}
%	\item \textbf{Matching Sessions}: Two sessions $\sid_{i}, \sid_j$ are matching if:
%	\begin{align*}
%		\owner[\sID_{i}] &\neq \owner[\sID_j] &&\text{(Different owners)}\\
%		\GakeUSet[\sID_{i}] &= \GakeUSet[\sID_j]  &&\text{(Identical participants)} \\
%		\changehighlight{\{\GIniMes[\sID_{i}]\}} \cup \GIniMesSet[\sID_{i}] &= \changehighlight{\{\GIniMes[\sID_j]\}} \cup \GIniMesSet[\sID_j]   &&\text{(Identical messages in the first round)}\\
%		\changehighlight{\{\GResMes[\sID_{i}]\}} \cup \GResMesSet[\sID_{i}] &= \changehighlight{\{\GResMes[\sID_j]\}}\cup \GResMesSet[\sID_j]  &&\text{(Identical messages in the second round)}
%	\end{align*}
%	%	$\owner[\sID_{i}] \neq \owner[\sID_j]$, $\GakeUSet[\sID_{i}] = \GakeUSet[\sID_j]$, $\GIniMes[\sID_{i}] \cup \GIniMesSet[\sID_{i}]= \GIniMes[\sID_j]\cup \GIniMesSet[\sID_j]$ and $\GResMes[\sID_{i}] \cup \GResMesSet[\sID_{i}]= \GResMes[\sID_j]\cup \GResMesSet[\sID_j]$. %(Notice that, we assume $\GakeUSet$ is an ordered set and $\User[j]$ represents the $j$-th party in $\GakeUSet$. The index $j$ is different from the party's index in the set of all participants $\PartySet$.)
%\end{itemize}
%\changehighlight{As in the $\AKE$ setting, our protocols in the full $\Gake$ model will use signatures, and hence any successful no-match attack as described in~\cite{CCS:LiSch17} will lead to a signature forgery.}
%%\todo{This is probably wrong and needs a fixup like we did for ake.}
%
%
%\heading{Test session.} The adversary is given access to the test oracle $\TestO$. This oracle can be queried multiple times and depending on a randomly chosen bit $b \getsr \bits$ (which is shared between all test queries), it outputs either a uniformly random key, or the specified session key.  
%
%
%%\todo{freshness+ attack table + formal def}
%%\owner[\sID_{i}] \neq \owner[\sID_j]$, $\GakeUSet[\sID_{i}] = \GakeUSet[\sID_j]$, $\GIniMes[\sID_{i}] = \GIniMes[\sID_j]$ and $\GResMes[\sID_{i}] = \GResMes[\sID_j]
%\begin{figure}[h!]
%	\centering
%	\hspace*{-0.1cm}\scalebox{0.92}{
%		\fbox{\small
%			\begin{minipage}[t]{12.7cm}
%				\underline{$\FreshnessO(\sID^*)$}
%				\begin{nicodemus}[1]
%					\item $({i}^\star, \GakeUSet^\star) := (\owner[\sID^*], \GakeUSet[\sID^*])$						
%					\item $\matchingSessions \coloneqq \{\sID \mid   \owner[\sID] \neq  {i}^\star~ \wedge ~ \GakeUSet[\sID]= \GakeUSet^\star ~$\\ 
%					\hspace*{3.2cm}$ \wedge~\changehighlight{\{\GIniMes[\sID]\}} \cup \GIniMesSet[\sID] = \changehighlight{\{\GIniMes[\sID^*]\}}\cup \GIniMesSet[\sID^*] ~$\\
%					\hspace*{3.2cm}$ \wedge~\changehighlight{\{\GResMes[\sID] \}}\cup \GResMesSet[\sID] = \changehighlight{\{\GResMes[\sID^*]\}}\cup \GResMesSet[\sID^*]\}$ \gcom{matching sessions}
%					\item \pcif $\revealed[\sID^*]$ \pcor $(\exists \sID \in \matchingSessions: \revealed[\sID] = \true)$
%					\item \quad \pcreturn $\pcfalse$ \gcom{$\Adv$ trivially learned the test session's key}
%					\item \pcif $\exists\sid\in\matchingSessions \suchthat \sid\in\testSessions$
%					\item \quad \pcreturn $\pcfalse$ \gcom{$\A$ also tested a matching session}
%					\item \pcreturn $\true$
%				\end{nicodemus}
%				\ \\
%				\underline{$\ValidAttackO(\sID^*)$}
%				\begin{nicodemus}
%					\item $({i}^\star, \GakeUSet^\star) := (\owner[\sID^*], \GakeUSet[\sID^*])$						
%					\item $\matchingSessions \coloneqq \{\sID \mid   \owner[\sID] \neq  {i}^\star~ \wedge ~ \GakeUSet[\sID]= \GakeUSet^\star ~$\\ 
%					\hspace*{3.2cm}$ \wedge~\changehighlight{\{\GIniMes[\sID]\}} \cup \GIniMesSet[\sID] = \changehighlight{\{\GIniMes[\sID^*]\}}\cup \GIniMesSet[\sID^*] ~$\\
%					\hspace*{3.2cm}$ \wedge~\changehighlight{\{\GResMes[\sID]\}} \cup \GResMesSet[\sID] = \changehighlight{\{\GResMes[\sID^*]\}}\cup \GResMesSet[\sID^*]\}$ \gcom{matching sessions}
%					\item $\pcfor \attack\in$\hspace*{-0.1em} \cref{tab:wFS-optimized-table-gake} 
%					\item \quad $\pcif \attack=\true~ \pcreturn \true$\label{line:attack-true-2-gake}
%					\item \pcreturn \pcfalse
%				\end{nicodemus}
%			\end{minipage}
%	}}
%	\caption{Helper procedures $\FreshnessO$ and $\ValidAttackO$ for game $\GAKEsecurityPFSGame$ defined in \cref{fig:game-ind-atk-gake}.	Procedure $\FreshnessO$ checks if the adversary performed some trivial attack. In procedure $\ValidAttackO$, each attack is evaluated by the set of variables shown in \cref{tab:wFS-optimized-table-gake} and checks if an allowed attack was performed. If the values of the variables are set as in the corresponding row, the attack was performed, i.\,e.\, $\attack=\true$, and thus the session is valid.
%	}
%	\label{fig:gake-freshness}
%\end{figure}
%
%
%\setlength\extrarowheight{1pt}
%\begin{table}[t]
%	\begin{center}
%		\scalebox{0.92}{
%			\begin{tabular}{|p{0.8cm}p{6cm}|C{.5cm}|C{1.2cm}|}
%				\hline
%				& \begin{minipage}[t][1cm][b]{5cm} $\A$ gets ($\owner[\sid^*], \PartySet{i} \coloneqq \peer[\sid^*]$)\end{minipage}
%				& \rotatebox{90}{\parbox{2.7cm}{$\peercorrupted[\sid^*]$}}& \rotatebox{90}{\parbox{2.5cm}{$|\matchingSessions|$}} \\
%				\hline\hline
%				%
%				0. & {\bf multiple matching sessions}  & --  & $  > |\PartySet{i}|$  \\
%				\hline
%				\hline
%				1. & {\bf (long-term, long-term)}   & --  &  $=|\PartySet{i}|$\\
%				\hline
%				\hline
%				2. & {\bf (long-term, long-term)}  & \tabfalse & $ < |\PartySet{i}|$  \\
%				\hline
%			\end{tabular}
%		}
%	\end{center}
%	\caption{Table of attacks for adversaries against explicitly authenticated group key exchange protocols without ephemeral state reveals. An attack is regarded as an AND conjunction of variables with specified values as shown in the each line, where “--” means that this variable can take arbitrary value and {\tabfalse} means “false”.}
%	\label{tab:wFS-optimized-table-gake}



\subsection{Hash Proof System}\label{subsec:HPF}
We now recall the definitions of smooth projective hashing and hash proof system, introduced for the first time in \cite{CS02} and later picked up in \cite{JKRS20}.

Informally, a projective hash function is a keyed hash function associated with two types of keys: the secret \textit{hashing key}, which allows for hashing of every element in the domain, and a public \textit{projective key}, that can be used to hash only those elements that lie in a designated subset (the language) of the domain.
A projective hash function is \textit{smooth} if the projective key gives only a negligible advantage in correctly hashing an element that lies outside the designated subset.

\begin{definition}[Smooth projective hashing]
	Let $\Domain,\Codomain$ be two sets, let $\Language \subset \Domain$ be a language and let $\SKSpace,\PKSpace$ be the secret and public key spaces.
	For any secret key $\sk \in \SKSpace$, let $\Lambda_{\sk} : \Domain \rightarrow \Codomain$ be a keyed hash function.
	A hash function $\Lambda_{\sk}$ is \textbf{projective} if there exists a projection $\projectKey : \SKSpace \rightarrow \PKSpace$ such that the key $\pk = \projectKey(\sk)$ uniquely defines the action of $\Lambda_\sk$ on $\Language$: For every $\x \in \Language$, the digest $\y = \Lambda_\sk(\x)$ can be consistently computed from $\pk$ and $\x$.
	
	A projective hash function is \textbf{smooth} if, given any hashing key $\sk \getsr \SKSpace$ and its projection key $\pk = \projectKey(\sk)$, there exists only a negligible probability to successfully distinguish between samples from the distributions $D_1$ and $D_2$, where
	\[ D_1 := \{ (\x', \pk, \y) \mid \y = \Lambda_\sk(\x')) \}_{\x' \in \Domain\setminus\Language} \quad \text{and} \quad D_2 := \{(\x', \pk, \y) \mid \y \getsr \Codomain \}_{\x' \in \Domain\setminus\Language}\]
	Equivalently,\mattia{check this, I give a different definition here} let $\adv$ be an adversary that tries to compute the action of $\HPSHash_\sk$ on $\Domain \setminus \Language$ using the projection key $\pk = \projectKey(\sk)$ of a random $\sk \getsr \SKSpace$; then, for any $\x' \in \Domain \setminus \Language$,
	\[ \Pr \left[ \y \gets \adv(\pk,\x')  \mid \y = \HPSHash_\sk(\x') \right] \leq \negl[\secpar]\]\mattia{Possibly add the definition of $k$-entropic}
\end{definition}

Before moving on with the definition of a hash proof system, a few assumptions need to be made.
First, we assume that the projection function $\projectKey$ and the algorithms for sampling elements from $\Domain$ and from $\Language$ are efficient.
Secondly, we assume that for every $\x \in \Language$, one can efficiently produce a witness $w$ proving that $\x$ belongs to the language.

\begin{definition}
	A \textbf{Hash Proof System} (HPS) consists of three algorithms $\HPS = (\HPSParam,\HPSPrivEval,\HPSPubEval)$ defined as follows:
	\begin{itemize}
		\item $\HPSparams: = (\HPSgroup,\Codomain,\Domain,\Language,\SKSpace,\PKSpace,\HPSHash,\projectKey)  \getsr \HPSParam (1^\secpar)$: a PPT parameter setup algorithm that outputs a parameter set $\HPSparams$ containing a possibly empty set $\HPSgroup$ of extra parameters, a codomain $\Codomain$, a domain $\Domain$, a language $\Language$, a hashing key space $\SKSpace$, a projective key space $\PKSpace$, a smooth projective hash function $\HPSHash_{(\cdot)}: \Domain \rightarrow \Codomain$ and a projection function $\projectKey: \SKSpace \rightarrow \PKSpace$;
		\item $\y \gets \HPSPrivEval (\sk,\x)$: a deterministic private hashing algorithm that, on input a secret hashing key $\sk$ and an element $\x \in \Language$, outputs the digest $\y = \HPSHash_\sk(\x)$;
		\item $\y \gets \HPSPubEval (\pk,\x,w)$: a deterministic public hashing algorithm that, on input a projection key $\pk = \mu(\sk)$, an element $\x \in \Language$ and a witness $w$ for the fact that $\x \in \Language$, outputs the digest $\y = \HPSHash_\sk(\x)$.
	\end{itemize}
\end{definition}

The fundamental problem which an HPS bases its security on is the $m$- fold subset membership problem, which asks to tell elements sampled from the language apart from elements sampled in its complement.
\begin{problem}[$m$-fold Subset Membership Problem]
	Given $m$ elements uniformly drawn from $\Language$ and $m$ elements uniformly drawn from $\Domain \setminus \Language$, an adversary $\adv$ has a negligible advantage in distinguishing them; more formally,
	\begin{align*} \AdvSMP :=  & \big| \Pr \left[ 1 \gets \adv(\Domain,\Language,\x_1,\dots,\x_m )  \mid \x_1,\dots,\x_m \getsr \Language \right] - \\
		&\Pr \left[ 1 \gets \adv(\Domain,\Language,\x_1',\dots,\x_m' )  \mid \x_1',\dots,\x_m' \getsr \Domain \setminus \Language \right] \big| \leq \negl[\secpar] 
		\end{align*}
\end{problem}



	\section{Protocol}\label{sec:protocol}
In this section we present our first protocol \mattia{name of the protocol}, focusing on a session run from the viewpoint of party $\User[i]$.
We then show its correctness, expanding the computation that $\User[i]$ performs in the key-derivation phase.
We prove the protocol secure \mattia{security notion and model} in \Cref{sec:security}.

\begin{figure}
	\begin{adjustbox}{varwidth=0.9\textwidth,fbox,center,scale=0.92}
	\begin{subfigure}[t]{.45\linewidth}
		\underline{$\GAKEIni(\PartySet_i)$:}\smallskip
		\begin{nicodemus}[1]
			\item \pcfetch $(S_j)_{\User[j]\in \PartySet_i}$
			\item $(\esk_i,\epk_i) \getsr \KeyGen(1^\secpar)$  
			\item $m_i \gets (\User[i],\epk_i)$
			\item \pcbroadcast $m_i$
		\end{nicodemus}
	\end{subfigure}%
	\begin{subfigure}[t]{.5\linewidth}
		\underline{$\GAKERes((\spk_j)_{\User[j] \in \PartySet_i},\GakeIniMesSet[i],\ssk_i,\esk_i)$:} \smallskip
		\begin{nicodemus}
			\item $\tempsek{i} \gets \left( \dfrac{\epk_{i+1}}{\epk_{i-1}}\right)^{\ssk_i}$ 	
			\item $\tempesk{i} \gets \left( \dfrac{\spk_{i+1}}{\spk_{i-1}}\right)^{\esk_i}$ 	
			\item $\tempeek{i} \gets \left( \dfrac{\epk_{i+1}}{\epk_{i-1}}\right)^{\esk_i}$ 	
			\item $\hat{m}_i \gets (\User[i],\tempsek{i},\tempesk{i},\tempeek{i})$ 
			\item \pcbroadcast $\hat{m}_i$ \\
		\end{nicodemus}
	\end{subfigure}
	\begin{subfigure}{0.7\linewidth}
		\underline{$\GAKEDer(\GakeResMesSet[i])$:} \smallskip
		\begin{nicodemus}
			\item $\groupsek{i} \gets E_{i-1}^{ns_i} \cdot (\tempsek{i})^{n-1} \cdot (\tempsek{i+1})^{n-2} \cdots \tempsek{i-2}$  \\
			\item $\groupesk{i} \gets S_{i-1}^{n\esk_i} \cdot (\tempesk{i})^{n-1} \cdot (\tempesk{i+1})^{n-2} \cdots \tempesk{i-2}$  \\
			\item $\groupeek{i} \gets E_{i-1}^{n\esk_i} \cdot (\tempeek{i})^{n-1} \cdot (\tempeek{i+1})^{n-2} \cdots \tempeek{i-2}$  \\
			\item $\context = ((\spk_j)_{\User[j] \in \PartySet},(\epk_j)_{\User[j] \in \PartySet})$
			\item $K \gets \FSHash (\context \concat \groupsek{i} \cdot \groupesk{i} \concat \groupeek{i})$\\
		\end{nicodemus}
	\end{subfigure}
	\end{adjustbox}
	\caption{The local session run by party $\User[i]$, with static key-pair $(\ssk_i, \spk_i)$. Recall that $\GakeIniMesSet[i]$ is the set of messages collected by $\User[i]$ at the end of the Initialization phase, while $\GakeResMesSet[i]$ is the set of those collected at the end of the Response phase.}
	\label{fig:ourprotocol}
\end{figure}

In the Setup phase, each party $\User[i]$ generates a static key pair $(\ssk_i,\spk_i)$, and registers its public key $\spk_i$ using a PKI.
In the Initialization phase, each party $\User[i]$ retrieves from the PKI the static keys $(\spk_1,\dots,\spk_n)$ of all the participants $\User[1],\dots,\User[n]$  in the key-exchange; then, it generates an ephemeral key pair $(\esk_i,\epk_i)$, and broadcasts the public key $\epk_i$. 
In the Response phase, each party mixes static and ephemeral key material:
%\begin{itemize}
%	\item with the subscript ``es'' we indicate the use of a static secret key on an ephemeral element
%	\item with the subscript ``se'' we indicate the use of an ephemeral secret key on a static element
%	\item with the subscript ``ee'' we indicate the use of an ephemeral secret key on an ephemeral element
%\end{itemize}
\begin{itemize}
	\item with $\tempsek{}$ we indicate the use of a static secret key on an ephemeral element
	\item with $\tempesk{}$ we indicate the use of an ephemeral secret key on a static element
	\item with $\tempeek{}$ we indicate the use of an ephemeral secret key on an ephemeral element
\end{itemize}
%The $\hat{ }$ symbol characterizes the intermediate key elements produced in the Response phase.
In the Key-derivation phase, each party computes the three partial keys $\groupsek{i}$, $\groupesk{i}$, $\groupeek{i}$ as in \Cref{fig:ourprotocol}, and multiplies the first two together to obtain the static-ephemeral key (the need for this step is explained in the proof of correctness).
Finally, the session key is computed as $K \gets \FSHash (\context \concat \groupsek{i} \cdot \groupesk{i} \concat \groupeek{i})$.

\medskip
\textbf{Correctness}. We show that each party $\User[i]$ successfully computes the group elements $g^{e_1s_2 + \dots e_ns_1}\cdot g^{s_1e_2 + \dots s_ne_1} = \groupsek{i} \cdot \groupesk{i}$ and $g^{e_1e_2 + \dots e_ne_1} = \groupeek{i}$.
We start off by explicitly expanding the partial key $\groupeek{i}$:
\begin{align*}
	\groupeek{i} &\gets E_{i-1}^{n\esk_i} \cdot (\tempeek{i})^{n-1} \cdot (\tempeek{i+1})^{n-2} \cdots \tempeek{i-2}\\
	&= E_{i-1}^{n\esk_i} \cdot  \dfrac{\epk_{i+1}^{(n-1)\esk_i}}{\epk_{i-1}^{(n-1)\esk_i}} \cdot \dfrac{\epk_{i+2}^{(n-2)\esk_{i+1}}}{\epk_{i}^{(n-2)\esk_{i+1}}} \cdots  \dfrac{\epk_{i-1}^{\esk_{i-2}}}{\epk_{i-3}^{\esk_{i-2}}}  \\
	&= E_{i-1}^{\esk_i} \cdot  \epk_{i+1}^{\esk_i} \cdot \epk_{i+2}^{\esk_{i+1}} \cdots  \epk_{i-1}^{\esk_{i-2}}  \\
	&= g^{\esk_{i-1}\esk_i} \cdot  g^{\esk_i\esk_{i+1}} \cdot g^{\esk_{i+1}\esk_{i+2}} \cdots  g^{\esk_{i-2}\esk_{i-1}}  \\
\end{align*}
As shown below, when dealing with the other partial keys  $\groupsek{i}$ and $\groupesk{i}$  individually, we cannot perform such neat simplifications but for the first term:

\begin{align*}
	\groupsek{i} &= \epk_{i-1}^{n\ssk_i} \cdot (\tempsek{i})^{n-1} \cdot (\tempsek{i+1})^{n-2} \cdots \tempsek{i-2} \\
	&= \epk_{i-1}^{n\ssk_i} \cdot  \dfrac{\epk_{i+1}^{(n-1)\ssk_i}}{\epk_{i-1}^{(n-1)\ssk_i}} \cdot \dfrac{\epk_{i+2}^{(n-2)\ssk_{i+1}}}{\epk_{i}^{(n-2)\ssk_{i+1}}} \cdots  \dfrac{\epk_{i-1}^{\ssk_{i-2}}}{\epk_{i-3}^{\ssk_{i-2}}}  \\
	&= g^{\esk_{i-1}\ssk_i} \cdot \epk_{i+1}^{(n-1)\ssk_i} \cdot \dfrac{\epk_{i+2}^{(n-2)\ssk_{i+1}}}{\epk_{i}^{(n-2)\ssk_{i+1}}} \cdots  \dfrac{\epk_{i-1}^{\ssk_{i-2}}}{\epk_{i-3}^{\ssk_{i-2}}}  \\
	\groupesk{i} &= \spk_{i-1}^{n\esk_i} \cdot (\tempesk{i})^{n-1} \cdot (\tempesk{i+1})^{n-2} \cdots \tempesk{i-2}  \\
	&= \spk_{i-1}^{n\esk_i} \cdot  \dfrac{\spk_{i+1}^{(n-1)\esk_i}}{\spk_{i-1}^{(n-1)\esk_i}} \cdot \dfrac{\spk_{i+2}^{(n-2)\esk_{i+1}}}{\spk_{i}^{(n-2)\esk_{i+1}}} \cdots  \dfrac{\spk_{i-1}^{\esk_{i-2}}}{\spk_{i-3}^{\esk_{i-2}}}  \\
	&= g^{\ssk_{i-1}\esk_i} \cdot \spk_{i+1}^{(n-1)\esk_i} \cdot \dfrac{\spk_{i+2}^{(n-2)\esk_{i+1}}}{\spk_{i}^{(n-2)\esk_{i+1}}} \cdots  \dfrac{\spk_{i-1}^{\esk_{i-2}}}{\spk_{i-3}^{\esk_{i-2}}}  \\
\end{align*}
It is easy to see that the missing simplifications are re-introduced once we multiply the partial keys together: note how the factor $\epk_{i+1}^{(n-1)\ssk_i}$ in $\groupsek{i}$ absorbs the denominator ${\spk_{i}^{(n-2)\esk_{i+1}}}$ in $\groupesk{i}$, resulting in $g^{\ssk_i \esk_{i+1}}$.
We can perform analogous computations to the other enumerators, both in $\groupsek{i}$ (giving rise to all terms $g^{\ssk_{j-1}\esk_j}$ but for $j=i$, which is already in $\groupesk{i}$) and in $\groupesk{i}$ (giving rise to all terms $g^{\esk_{j-1}\ssk_j}$ but for $j=i$, which is already in $\groupsek{i}$), and the commutativity of group multiplication concludes the proof.\medskip

\textbf{Security proof.}
{\color{red}Work in progress!!!}

\Game{$G_0$}
In this game we abort if the uniqueness of the session is broken, i.e. if two global sessions result in the same session key. 
We assume that each static public key is belongs a single user and acts like an identifier (a reasonable assumption, considering the exponential size of the key space and the polynomial size of the party set). 
For two session keys to be equal, the remaining elements of the context $\context$ must be the same.
This happens if and only if each party in $\PartySet$ samples the same ephemeral keys in two distinct sessions.
In the worst case scenario, where all $\maxusers$ parties are involved ($\PartySet = \Universe$), this happens with probability $\maxsessions^2/|\GG|$, where $\maxsessions$ is the maximum number of global sessions and $|\GG|$ is the size of the group.
Thus
\[ \prob{1 \gets G_0^\adv} \leq \dfrac{\maxsessions}{|\GG|}.\]
	\section{Security}\label{sec:security}
We show our protocol secure.
	\section{Conclusions}\label{sec:conclusions}
We achieve something great.
Deal with it and publish our paper.
	

	\bibliography{cryptobib/abbrev3,cryptobib/crypto,add}
%	\bibliography{abbrev0,bibliography}
	
\end{document}
